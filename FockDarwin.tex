\documentclass{article}
\usepackage{amsmath,amssymb}

\usepackage{ulem}
\input{notes-prelude.in}

\begin{document}

\section{Diagonalization}

Consider the case of $N$ bosons in a circular cross-section trap rotating with angular velocity $\Omega$.  Setting $\hbar=m=1$, the Hamiltonian for such a system is given by
$$H = \sum_{i=1}^N \left[\frac{p_{x_i}^2 + p_{y_i}^2 + p_{z_i}^2}{2} - \Omega(x_i
  p_{y_i} - y p_{x_i}) + \frac{\om^2}{2}(x^2+y^2) + \frac{\om_z^2}{2}z^2\right].$$

We define the ladder operators
$$
\begin{aligned}
&a_i:= \sqrt{\frac{\om}{2}}x_i + i\sqrt{\frac{1}{2\om}}p_{x_i},
&a_i^\dagger := \sqrt{\frac{\om}{2}}x_i - i\sqrt{\frac{1}{2\om}}p_{x_i},\\
&b_i:= \sqrt{\frac{\om}{2}}y_i + i\sqrt{\frac{1}{2\om}}p_{y_i},
&b_i^\dagger := \sqrt{\frac{\om}{2}}y_i - i\sqrt{\frac{1}{2\om}}p_{y_i} ,\\
&c_i:= \sqrt{\frac{\om_z}{2}}z_i + i\sqrt{\frac{1}{2\om_z}}p_{z_i},
&c_i^\dagger := \sqrt{\frac{\om_z}{2}}z_i - i\sqrt{\frac{1}{2\om_z}}p_{z_i} ,\\
\end{aligned}
$$
and then observe that using these operators we can re-write our
original Hamiltonian in the form,

$$
\begin{aligned}
H
&= \sum_{i=1}^N \left[\om\paren{a_i^\dagger a_i + b^\dagger b + 1} +  \om_z\paren{c_i^\dagger c_i + \half} - i\Omega\paren{a_i^\dagger b_i - b_i^\dagger a_i}\right],\\
\end{aligned}
$$
To diagonalize this, we let
$$
\begin{aligned}
&\gamma_i := \frac{a_i + i b_i}{\sqrt{2}}, \quad \eta_i := \frac{i a_i + b_i}{\sqrt{2}},
&\Rightarrow\quad a_i = \frac{\gamma_i - i \eta_i}{\sqrt 2}, \quad b_i =\frac{\eta_i - i \gamma_i}{\sqrt 2}.
\end{aligned}
$$
Note that
$$
\begin{aligned}
\left[\gamma_i,\gamma_i^\dagger\right] &=
  \frac{\left[a_i,a_i^\dagger\right] +
  \left[ib_i,-ib_i^\dagger\right]}{2} = \frac{1+1}{2} = 1,\\
\left[\eta_i,\eta_i^\dagger\right] &=
  \frac{\left[i a_i,-i a_i^\dagger\right] +
  \left[b_i,b_i^\dagger\right]}{2} = \frac{1+1}{2} = 1,
\end{aligned}
$$
so we see that these new ladder operators satisfy the usual
commutation relations.  Rewriting our Hamiltonian in terms of these
operators, we obtain
$$
\begin{aligned}
H
&=  \sum_{i=1}^N \left\{\om\left[\gamma_i^\dagger\gamma_i + \eta_i^\dagger\eta_i + 1\right] + \om_z\paren{c_i^\dagger c_i + \half}+ \Omega\left[\eta_i^\dagger\eta_i - \gamma_i^\dagger\gamma_i\right]\right\}
\end{aligned}
$$

In this form we see that the eigenspectrum of the Hamiltonian is given in terms of a (bosonic) ground state, $\ket{0}$, that is annihilated by $\gamma_i$, $\eta_i$ and $c_i$ for every $i=1\dots N$, and excited states obtained from the raising operators $\gamma_i^\dagger$, $\eta_i^\dagger$, and $c_i^\dagger$.

\section{Investigation of first excited state}

We now consider the first excited state of this system.  In order to preserve bosonic symmetry, we must use a symmetrized sum of the raising operators,
$$\xi := \frac{1}{\sqrt{N}}\sum_i \eta_i.$$  Observe that due to the normalization factor, we have that $[\xi,xi^\dagger] = 1$.

Define $\ket{1}:= \xi\ket{0}.$  We now want to examine the angular momentum of $\ket{1}$.  Observing that the position and momentum operators can be expressed in terms of the new ladder operators as follows,
$$
x_i = \frac{1}{2\sqrt{\om}}[\gamma_i^\dagger + \gamma_i+ i(\eta_i^\dagger-\eta_i)],
y_i = \frac{1}{2\sqrt{\om}}[\eta_i^\dagger + \eta_i+ i(\gamma_i^\dagger-\gamma_i)],
$$
$$
z_i = \sqrt{\frac{1}{2\om}}[c_i^\dagger + c_i],
$$
$$
p_{x_i} = i\frac{\sqrt{\om}}{2}[\gamma_i^\dagger - \gamma_i + i(\eta_i^\dagger + \eta_i)],
p_{y_i} =i\frac{\sqrt{\om}}{2}[\eta_i^\dagger - \eta_i + i(\gamma_i^\dagger +\gamma_i)],
$$
$$
p_{z_i} =i\sqrt{\frac{\om}{2}}[c_i^\dagger - c_i],
$$
we see that therefore
$$
J_{z_i}=x_i p_{y_i} - y_i p_{x_i} = \eta_i^\dagger\eta_i - \gamma_i^\dagger\gamma_i
$$
Define $J_{\text{total}}:=\sum_i J_{z_i}$.  Observe that $[J_{z_i},H]=0$ and thus $[J_{\text{total}},H]=0$, so that the total $z$ component of angular momentum for the system and the energy of the entire system can be known simultaneously.  Next observe that $[J_{z_i},\eta_{i'}^\dagger]=\eta^\dagger \delta_{ii'}$, so that $[J_{z_i},\xi^\dagger]=\eta_i^\dagger$ and $[J_{\text{total}},\xi^\dagger]=\xi^\dagger$.  Because of this, $\xi^\dagger$ does not create an eigenstate of $z$ angular momentum for any singular particle, but rather it creates an eigenstate of the whole system with total $z$ angular momentum of 1, as we can see from the fact that
$$J_{\text{total}} \xi^\dagger \ket{0} = \xi^\dagger J_{\text{total}} \ket{0} + \xi^\dagger \ket{0} = 1\cdot \xi^\dagger \ket{0}.$$
Likewise, since $[H,\xi^\dagger] = \xi^\dagger (\om+\Omega)$, we see that this state has energy equal to $\om+\Omega+\frac{3}{2}$.

Now, slightly abusing notation, we consider the state $\ket{N}:=\prod \eta_i^\dagger$.  

\end{document}
